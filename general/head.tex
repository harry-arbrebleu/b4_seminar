\usepackage{luatexja} % LuaTeXで日本語を使うためのパッケージ
\usepackage{luatexja-fontspec} % LuaTeX用の日本語フォント設定
% lualatex main && upmendex -r -c -s main.ist -g main && lualatex main  

% --- 数学関連 ---
\usepackage{amsmath, amssymb, amsfonts, mathtools, bm, amsthm} % 基本的な数学パッケージ
\usepackage{type1cm, upgreek} % 数式フォントとギリシャ文字k
\usepackage{physics, mhchem} % 物理や化学の記号や式の表記を簡単にする

% --- 表関連 ---
\usepackage{multirow, longtable, tabularx, array, colortbl, dcolumn, diagbox} % 表のレイアウトを柔軟にする
\usepackage{tablefootnote, truthtable} % 表中に注釈を追加、真理値表
\usepackage{tabularray} % 高度な表組みレイアウト

% --- グラフィック関連 ---
\usepackage{tikz, graphicx} % 図の描画と画像の挿入
% \usepackage{background} % ウォーターマークの設定
\usepackage{caption, subcaption} % 図や表のキャプション設定
\usepackage{float, here} % 図や表の位置指定

% --- レイアウトとページ設定 ---
\usepackage{fancyhdr} % ページヘッダー、フッター、余白の設定
\usepackage[top = 20truemm, bottom = 20truemm, left = 20truemm, right = 20truemm]{geometry}
\usepackage{fancybox, ascmac} % ボックスのデザイン

% --- 色とスタイル ---
\usepackage{xcolor, color, colortbl, tcolorbox} % 色とカラーボックス
\usepackage{listings, jvlisting} % コードの色付けとフォーマット

% --- 参考文献関連 ---
\usepackage{biblatex, usebib} % 参考文献の管理と挿入
\usepackage{url, hyperref} % URLとリンクの設定

% --- その他の便利なパッケージ ---
\usepackage{footmisc} % 脚注のカスタマイズ
\usepackage{multicol} % 複数段組
\usepackage{comment} % コメントアウトの拡張
\usepackage{siunitx} % 単位の表記
\usepackage{docmute, needspace}
\usepackage{makeidx, minted}

\sisetup{
  table-format = 1.5,
  table-number-alignment = center,
}
\newcounter{problem}[section]
\renewcommand*\oldstylenums[1]{\textosf{#1}}
% \abovedisplayskip = 0pt
% \belowdisplayskip = 0pt

\allowdisplaybreaks
\newcolumntype{t}{!{\vrule width 0.1pt}}
\newcolumntype{b}{!{\vrule width 1.5pt}}
\UseTblrLibrary{amsmath, booktabs, counter, diagbox, functional, hook, html, nameref, siunitx, varwidth, zref}
\setlength{\columnseprule}{0.4pt}
\captionsetup[figure]{font = bf}
\captionsetup[table]{font = bf}
\captionsetup[lstlisting]{font = bf}
\captionsetup[subfigure]{font = bf, labelformat = simple}
\setcounter{secnumdepth}{5}
\newcolumntype{d}{D{.}{.}{5}}
\newcolumntype{M}[1]{>{\centering\arraybackslash}m{#1}}
\everymath{\displaystyle}

\renewcommand{\figurename}{図}
\renewcommand{\tablename}{表}
\renewcommand{\theproblem}{\thesection-\arabic{problem}}
\renewcommand{\i}{\mathrm{i}}
\renewcommand{\laplacian}{\Delta}
\newcommand{\NOT}[1]{\overline{#1}}
\renewcommand{\hat}[1]{\overhat{#1}}
\renewcommand{\thesubfigure}{(\alph{subfigure})}
\newcommand{\m}[3]{\multicolumn{#1}{#2}{#3}}
\renewcommand{\r}[1]{\mathrm{#1}}
\newcommand{\e}{\r{e}}
\newcommand{\Ef}{E_{\r{F}}}
\renewcommand{\c}{\si{\degreeCelsius}}
\renewcommand{\d}{\r{d}}
\renewcommand{\t}[1]{\texttt{#1}}
\newcommand{\kb}{k_{\r{B}}}
\renewcommand{\phi}{\varphi}
\newcommand{\reff}[1]{\textbf{図\ref{#1}}}
\newcommand{\reft}[1]{\textbf{表\ref{#1}}}
\newcommand{\refe}[1]{\textbf{式\eqref{#1}}}
\newcommand{\refp}[1]{\textbf{ソースコード\ref{#1}}}
\newcommand{\refa}[1]{\textbf{\ref{#1}}}
\newcommand{\probbref}[1]{{\bfseries\sffamily 基礎問題\ref{problemb-ref:#1}}}
\newcommand{\probaref}[1]{{\bfseries\sffamily 発展問題\ref{problem-ref:#1}}}
\renewcommand{\lstlistingname}{ソースコード}
\renewcommand{\theequation}{\thesection.\arabic{equation}}
\renewcommand{\thefigure}{\thesection.\arabic{figure}}
\renewcommand{\thetable}{\thesection.\arabic{table}}
% \renewcommand{\thelstlisting}{\thesection.\arabic{lstlisting}}
% \renewcommand{\the}{def}
\renewcommand{\footrulewidth}{0.4pt}
\newcommand{\mar}[1]{\textcircled{\scriptsize #1}}
\newcommand{\combination}[2]{{}_{#1} \mathrm{C}_{#2}}
\newcommand{\thline}{\noalign{\hrule height 0.1pt}}
\newcommand{\bhline}{\noalign{\hrule height 1.5pt}}
\renewcommand{\thesection}{\arabic{section}}
\renewcommand{\appendixname}{付録}
\newcommand{\ketH}{\ket{\r{H}}}
\newcommand{\ketV}{\ket{\r{V}}}
\newcommand{\ketD}{\ket{\r{D}}}
\newcommand{\ketA}{\ket{\r{A}}}
\newcommand{\ketL}{\ket{\r{L}}}
\newcommand{\ketR}{\ket{\r{R}}}
\newcommand{\braH}{\bra{\r{H}}}
\newcommand{\braV}{\bra{\r{V}}}
\newcommand{\braD}{\bra{\r{D}}}
\newcommand{\braA}{\bra{\r{A}}}
\newcommand{\braL}{\bra{\r{L}}}
\newcommand{\braR}{\bra{\r{R}}}

\pagestyle{fancy}

\chead{B4新人研修}
\rhead{}
\cfoot{\thepage}
\lhead{}
\rfoot{62200139\ 青木\ 陽}
\setcounter{tocdepth}{4}
\makeatletter
\@addtoreset{equation}{subsection}
\makeatother

\title{偏光を用いた2量子もつれの量子状態トモグラフィによる評価}
\date{\today}
\author{62200139\ 青木\ 陽}
\addbibresource{ref.bib}
\defbibheading{bunken}[\refname]{\section*{#1}}