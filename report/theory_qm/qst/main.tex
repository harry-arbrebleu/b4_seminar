\documentclass{report}
\usepackage{luatexja}
\usepackage{tikz}
\usepackage{amsmath, amssymb, type1cm, amsfonts, latexsym, mathtools, bm, amsthm, url, color}
\usepackage{multirow, hyperref, longtable, dcolumn, tablefootnote, empheq}
\usepackage{tabularx, footmisc, colortbl, here, usebib, microtype, physics}
\usepackage{graphicx, luacode, caption, fancyhdr, appendix}
\usepackage{fancybox, color, tcolorbox, physics}
\usepackage[top = 20truemm, bottom = 20truemm, left = 20truemm, right = 20truemm]{geometry}
\usepackage{ascmac, fancybox, color, tabularray, subcaption}
\usepackage{multicol, siunitx}
\usepackage{luatexja-fontspec, ascmac, fancybox, pdfpages}
\usepackage{upgreek, colortbl, mhchem}
\usepackage{biblatex, array, truthtable}
\usepackage{listings, jvlisting}
\usepackage{xcolor, siunitx, float, dcolumn}
\usepackage{titlesec}
\usepackage{minted}

\titleformat{\section}[block]{\normalsize}{\LARGE{\textbf{第\thesection 回授業}}}{0.5em}{}
\sisetup{
  table-format = 1.5,
  table-number-alignment = center,
}
\newcounter{problem}[section]
\renewcommand*\oldstylenums[1]{\textosf{#1}}
% \abovedisplayskip = 0pt
% \belowdisplayskip = 0pt

\allowdisplaybreaks
\newcolumntype{t}{!{\vrule width 0.1pt}}
\newcolumntype{b}{!{\vrule width 1.5pt}}
\UseTblrLibrary{amsmath, booktabs, counter, diagbox, functional, hook, html, nameref, siunitx, varwidth, zref}
\setlength{\columnseprule}{0.4pt}
\captionsetup[figure]{font = bf}
\captionsetup[table]{font = bf}
\captionsetup[lstlisting]{font = bf}
\captionsetup[subfigure]{font = bf, labelformat = simple}
\setcounter{secnumdepth}{5}
\newcolumntype{d}{D{.}{.}{5}}
\newcolumntype{M}[1]{>{\centering\arraybackslash}m{#1}}
\everymath{\displaystyle}

\renewcommand{\figurename}{図}
\renewcommand{\tablename}{表}
\renewcommand{\theproblem}{\thesection-\arabic{problem}}
\renewcommand{\i}{\mathrm{i}}
\renewcommand{\laplacian}{\Delta}
\newcommand{\NOT}[1]{\overline{#1}}
\renewcommand{\hat}[1]{\overhat{#1}}
\renewcommand{\thesubfigure}{(\alph{subfigure})}
\newcommand{\m}[3]{\multicolumn{#1}{#2}{#3}}
\renewcommand{\r}[1]{\mathrm{#1}}
\newcommand{\e}{\r{e}}
\newcommand{\Ef}{E_{\r{F}}}
\renewcommand{\c}{\si{\degreeCelsius}}
\renewcommand{\d}{\r{d}}
\renewcommand{\t}[1]{\texttt{#1}}
\newcommand{\kb}{k_{\r{B}}}
\renewcommand{\phi}{\varphi}
\newcommand{\reff}[1]{\textbf{図\ref{#1}}}
\newcommand{\reft}[1]{\textbf{表\ref{#1}}}
\newcommand{\refe}[1]{\textbf{式\eqref{#1}}}
\newcommand{\refp}[1]{\textbf{ソースコード\ref{#1}}}
\newcommand{\refa}[1]{\textbf{\ref{#1}}}
\newcommand{\probbref}[1]{{\bfseries\sffamily 基礎問題\ref{problemb-ref:#1}}}
\newcommand{\probaref}[1]{{\bfseries\sffamily 発展問題\ref{problem-ref:#1}}}
\renewcommand{\lstlistingname}{ソースコード}
\renewcommand{\theequation}{\thesection.\arabic{equation}}
\renewcommand{\thefigure}{\thesection.\arabic{figure}}
\renewcommand{\thetable}{\thesection.\arabic{table}}
% \renewcommand{\thelstlisting}{\thesection.\arabic{lstlisting}}
% \renewcommand{\the}{def}
\renewcommand{\footrulewidth}{0.4pt}
\newcommand{\mar}[1]{\textcircled{\scriptsize #1}}
\newcommand{\combination}[2]{{}_{#1} \mathrm{C}_{#2}}
\newcommand{\thline}{\noalign{\hrule height 0.1pt}}
\newcommand{\bhline}{\noalign{\hrule height 1.5pt}}
\renewcommand{\thesection}{\arabic{section}}
\renewcommand{\appendixname}{付録}

\pagestyle{fancy}

\chead{B4新人研修}
\rhead{}
\cfoot{\thepage}
\lhead{}
\rfoot{62200139\ 青木\ 陽}
\setcounter{tocdepth}{4}
\makeatletter
\@addtoreset{equation}{subsection}
\makeatother

\title{偏光を用いた2量子もつれの量子状態トモグラフィによる評価}
\date{\today}
\author{62200139\ 青木\ 陽}
\addbibresource{ref.bib}
\defbibheading{bunken}[\refname]{\section*{#1}}
\begin{document}
  \subsection{密度行列}
    密度行列$\hat{\rho}$は状態$\ket{\psi}$に対して,
    \begin{align}
      \hat{\rho} \coloneqq \ket{\rho}\bra{\rho}
    \end{align}
    と定義される.
    任意の物理量$\hat{A}$の期待値が$\tr\qty(\hat{A}\hat{\rho})$と書けることを示す.
    \begin{proof}
      $\ket{\psi}$の存在するHilbert空間の基底全体を$\qty{\ket{n}}$とする.
      \begin{align}
        \ev{A} &= \mel**{\psi}{\hat{A}}{\psi} \\ 
        &= \mel**{\psi}{\hat{1}\hat{A}\hat{1}}{\psi} \\ 
        &= \sum_{n, n'} \braket{\psi}{x}\mel**{x}{\hat{A}}{x'}\braket{x'}{\psi} \\ 
        &= \sum_{n, n'} \mel**{x}{\hat{A}}{x'} \braket{\psi}{x}\braket{x'}{\psi} \\ 
        &= \sum_{n} \mel**{x}{\hat{A}\hat{\rho}}{x} \\ 
        &= \tr\qty{\hat{A}\hat{\rho}}\label{expatation-value-density-matrix}
      \end{align}
    \end{proof}
  \subsection{偏光基底}
    さて,前小節までで量子状態トモグラフィーを議論するための道具がそろった.
    以降では,具体的な偏光を用いた量子状態を再現する方法を議論する. % ここで偏光状態を基底とできる根拠(古典光学との繋がりで,Jones ベクトルなど)
    以降で用いる種々の偏光ベクトルを定義しておく.
    \begin{align}
      \ketH &\coloneqq \mqty(1 \\ 0) \\ 
      \ketV &\coloneqq \mqty(0 \\ 1) \\ 
      \ketD &\coloneqq \frac{1}{\sqrt{2}}\mqty(1 & 1) \\ 
      \ketD &\coloneqq \frac{1}{\sqrt{2}}\mqty(1 & -1) \\ 
      \ketL &\coloneqq \frac{1}{\sqrt{2}}\mqty(1 & \i) \\ 
      \ketR &\coloneqq \frac{1}{\sqrt{2}}\mqty(1 & -\i)
    \end{align}
    また,今後の計算のために,自分自身との外積を計算しておく.
    \begin{align}
      \ketH\braH &= \mqty(1 & 0 \\ 0 & 0) \label{outer-product-h} \\ 
      \ketV\braV &= \mqty(0 & 0 \\ 0 & 1) \label{outer-product-v} \\ 
      \ketD\braD &= \frac{1}{2}\mqty(1 & 1 \\ 1 & 1) \label{outer-product-d} \\ 
      \ketA\braA &= \frac{1}{2}\mqty(1 & -1 \\ -1 & 1) \label{outer-product-a} \\ 
      \ketL\braL &= \frac{1}{2}\mqty(1 & \i \\ -\i & 1) \label{outer-product-l} \\ 
      \ketR\braR &= \frac{1}{2}\mqty(1 & -\i \\ \i & 1) \label{outer-product-r}
    \end{align}
    さて,$\ketH$と$\ketV$は直交しているので,任意の量子状態$\ket{\psi}$は水平偏光$\ketH$と垂直偏光$\ketV$を基底として,
    \begin{align}
      \ket{\psi} = c_{\r{H}}\ketH + c_{\r{V}}\ketV \quad \abs{c_1}^2 + \abs{c_2}^2 = 1
    \end{align}
    と書ける.
    また,このときの密度行列$\hat{\rho}$は,
    \begin{align}
      \hat{\rho} &= \ket{\psi}\bra{\psi} \\ 
      &= \qty(c_{\r{H}}\ketH + c_{\r{V}}\ketV)\qty(c^*_{\r{H}}\braH + c^*_{\r{V}}\braV) \\ 
      &= \abs{c_{\r{H}}}^2 \ketH\braH + c_{\r{V}}c^*_{\r{H}}\braV\ketH + c_{\r{H}}c^*_{\r{V}}\braH\ketV + \abs{c_{\r{V}}}\ketV\braV \\ 
      &= \mqty(\abs{c_{\r{H}}}^2 & c_{\r{H}}c^*_{\r{V}} \\ c_{\r{V}}c^*_{\r{H}} & \abs{c_{\r{V}}}^2) \label{stete-vector-1-bit-hv-representation}
    \end{align}
    と書ける.
    これは,密度行列が分かれば量子状態を推定することができることを意味している.
    以降では,具体的な状態ベクトルを直接再現するよりも,多体系を考えたときにより豊かな表現力をもつ密度行列で考える.
  \subsection{密度行列の再構成}
    さて,実際の実験系で密度行列を再構成する方法について考える.
    この方法を量子状態トモグラフィーという.
    我々行うことのできる射影測定は,偏光の測定である.
    まずは1量子の再構成について述べる.
    \refe{stete-vector-1-bit-hv-representation}を求めたときのように$\ketH\braV$のような項を再構成することは困難であるため,
    密度行列のHermite性と,Pauli行列を$\sqrt{2}$で割ったものが正規直交基底をなすことを用いると,
    状態ベクトルを再構成することと,
    \begin{align}
      \hat{\rho} = \sum_{i = 0}^{3}u_i\hat{\sigma}^i
    \end{align}
    の$\qty{u_i}$を求めることは等価である.
    さらに,$\hat{\sigma}^i$はHermite演算子の基底になっていることから,
    係数$u_i$は,
    \begin{align}
      u_i &= \qty(\hat{\sigma}^i, \hat{\rho}) \\ 
      &= \tr{\qty(\hat{\sigma}^i)^{\dag}\hat{\rho}} \\ 
      &= \tr{\hat{\sigma}^i\hat{\rho}} \\ 
      &= \ev{\sigma^i} \label{1-bit-qst-u}
    \end{align}
    と変形できる.
    最後の式変形で\refe{expatation-value-density-matrix}を用いた.
    \refe{1-bit-qst-u}の結果は,我々に物理量$\sigma^i$の期待値が分かれば,密度行列を再構成することができることを教える.
    ところが,$\sigma^i$を直接測定することはできないため,偏光を用いて表す.
    \refe{outer-product-h}から\refe{outer-product-r}を用いると,
    \begin{align}
      \hat{\sigma}^0 &= \ketH\braH + \ketV\braV \label{sigma-0-uses-polarization} \\ 
      \hat{\sigma}^1 &= \ketD\braD - \ketA\braA = 2\ketD\braD - \hat{\sigma}^0 \label{sigma-1-uses-polarization} \\ 
      \hat{\sigma}^2 &= \ketR\braR - \ketL\braL = 2\ketR\braR - \hat{\sigma}^0 \label{sigma-2-uses-polarization} \\ 
      \hat{\sigma}^3 &= \ketH\braH - \ketV\braV\label{sigma-3-uses-polarization}
    \end{align}
    となる.
    \refe{sigma-0-uses-polarization}から\refe{sigma-3-uses-polarization}より,結局,我々は水平偏光,垂直偏光,45度偏光,右回り偏光の4種類を測定すれば良いことが分かる.
    それらの測定の期待値を$N_{\r{H}}$,$N_{\r{V}}$,$N_{\r{D}}$,$N_{\r{R}}$とすると,
    \begin{align}
      \hat{\rho} &= \frac{1}{\sqrt{2}}\qty[(N_{\r{H}} + N_{\r{V}})\hat{\sigma}^0 + (2N_{\r{D}} - N_{\r{H}} - N_{\r{V}})\hat{\sigma}^1 + (2N_{\r{R}} - N_{\r{H}} - N_{\r{V}})\hat{\sigma}^2 + (N_{\r{H}} - N_{\r{V}})\hat{\sigma}^3]
    \end{align}
    とすればよい.
    \par
    同様に2量子の場合も検討することができる.
    この計算は非常に煩雑であるため,結果は省略するが,$\sigma^i\otimes\sigma^j$をHilbert空間の基底として展開すればよい.
    実際には計算機で計算を行い,そのプログラムを付録にて示す.
\end{document}