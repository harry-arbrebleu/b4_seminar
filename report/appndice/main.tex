\documentclass{report}
\usepackage{luatexja}
\usepackage{tikz}
\usepackage{amsmath, amssymb, type1cm, amsfonts, latexsym, mathtools, bm, amsthm, url, color}
\usepackage{multirow, hyperref, longtable, dcolumn, tablefootnote, empheq}
\usepackage{tabularx, footmisc, colortbl, here, usebib, microtype, physics}
\usepackage{graphicx, luacode, caption, fancyhdr, appendix}
\usepackage{fancybox, color, tcolorbox, physics}
\usepackage[top = 20truemm, bottom = 20truemm, left = 20truemm, right = 20truemm]{geometry}
\usepackage{ascmac, fancybox, color, tabularray, subcaption}
\usepackage{multicol, siunitx}
\usepackage{luatexja-fontspec, ascmac, fancybox, pdfpages}
\usepackage{upgreek, colortbl, mhchem}
\usepackage{biblatex, array, truthtable}
\usepackage{listings, jvlisting}
\usepackage{xcolor, siunitx, float, dcolumn}
\usepackage{titlesec}
\usepackage{minted}

\titleformat{\section}[block]{\normalsize}{\LARGE{\textbf{第\thesection 回授業}}}{0.5em}{}
\sisetup{
  table-format = 1.5,
  table-number-alignment = center,
}
\newcounter{problem}[section]
\renewcommand*\oldstylenums[1]{\textosf{#1}}
% \abovedisplayskip = 0pt
% \belowdisplayskip = 0pt

\allowdisplaybreaks
\newcolumntype{t}{!{\vrule width 0.1pt}}
\newcolumntype{b}{!{\vrule width 1.5pt}}
\UseTblrLibrary{amsmath, booktabs, counter, diagbox, functional, hook, html, nameref, siunitx, varwidth, zref}
\setlength{\columnseprule}{0.4pt}
\captionsetup[figure]{font = bf}
\captionsetup[table]{font = bf}
\captionsetup[lstlisting]{font = bf}
\captionsetup[subfigure]{font = bf, labelformat = simple}
\setcounter{secnumdepth}{5}
\newcolumntype{d}{D{.}{.}{5}}
\newcolumntype{M}[1]{>{\centering\arraybackslash}m{#1}}
\everymath{\displaystyle}

\renewcommand{\figurename}{図}
\renewcommand{\tablename}{表}
\renewcommand{\theproblem}{\thesection-\arabic{problem}}
\renewcommand{\i}{\mathrm{i}}
\renewcommand{\laplacian}{\Delta}
\newcommand{\NOT}[1]{\overline{#1}}
\renewcommand{\hat}[1]{\overhat{#1}}
\renewcommand{\thesubfigure}{(\alph{subfigure})}
\newcommand{\m}[3]{\multicolumn{#1}{#2}{#3}}
\renewcommand{\r}[1]{\mathrm{#1}}
\newcommand{\e}{\r{e}}
\newcommand{\Ef}{E_{\r{F}}}
\renewcommand{\c}{\si{\degreeCelsius}}
\renewcommand{\d}{\r{d}}
\renewcommand{\t}[1]{\texttt{#1}}
\newcommand{\kb}{k_{\r{B}}}
\renewcommand{\phi}{\varphi}
\newcommand{\reff}[1]{\textbf{図\ref{#1}}}
\newcommand{\reft}[1]{\textbf{表\ref{#1}}}
\newcommand{\refe}[1]{\textbf{式\eqref{#1}}}
\newcommand{\refp}[1]{\textbf{ソースコード\ref{#1}}}
\newcommand{\refa}[1]{\textbf{\ref{#1}}}
\newcommand{\probbref}[1]{{\bfseries\sffamily 基礎問題\ref{problemb-ref:#1}}}
\newcommand{\probaref}[1]{{\bfseries\sffamily 発展問題\ref{problem-ref:#1}}}
\renewcommand{\lstlistingname}{ソースコード}
\renewcommand{\theequation}{\thesection.\arabic{equation}}
\renewcommand{\thefigure}{\thesection.\arabic{figure}}
\renewcommand{\thetable}{\thesection.\arabic{table}}
% \renewcommand{\thelstlisting}{\thesection.\arabic{lstlisting}}
% \renewcommand{\the}{def}
\renewcommand{\footrulewidth}{0.4pt}
\newcommand{\mar}[1]{\textcircled{\scriptsize #1}}
\newcommand{\combination}[2]{{}_{#1} \mathrm{C}_{#2}}
\newcommand{\thline}{\noalign{\hrule height 0.1pt}}
\newcommand{\bhline}{\noalign{\hrule height 1.5pt}}
\renewcommand{\thesection}{\arabic{section}}
\renewcommand{\appendixname}{付録}

\pagestyle{fancy}

\chead{B4新人研修}
\rhead{}
\cfoot{\thepage}
\lhead{}
\rfoot{62200139\ 青木\ 陽}
\setcounter{tocdepth}{4}
\makeatletter
\@addtoreset{equation}{subsection}
\makeatother

\title{偏光を用いた2量子もつれの量子状態トモグラフィによる評価}
\date{\today}
\author{62200139\ 青木\ 陽}
\addbibresource{ref.bib}
\defbibheading{bunken}[\refname]{\section*{#1}}
\begin{document}
  \chapter{数学の関係式}\label{pauli-matrix}
    ここでは,ベクトル空間と内積空間についての定義を与えたあと,Hermite行列が$\r{Herm}(N)$が内積空間となることを示す.
    ただし,Hermite行列は,
    \begin{align}
      \r{Herm}(N) \in \qty{\hat{H} \in \mathbb{C}^{N\times N} \mid \hat{H} = \hat{H}^{\dag}}
    \end{align}
    である.
    \par
    $V$が$K$上のベクトル空間であることは,以下の条件を全て満たすことである.
    ただし,$\bm{u}, \bm{v}, \bm{w} \in V$,$a, b \in K$とする.
    \begin{itembox}[l]{ベクトル空間の定義}
      \begin{enumerate}
        \item $\forall \bm{u}, \bm{v}, \bm{w} \in V\ \bm{u} + (\bm{v} + \bm{v}) = (\bm{u} + \bm{v}) + \bm{v}$
        \item $\exists \bm{0} \in V\ \forall \bm{v}\ \bm{u} + \bm{0} = \bm{0} + \bm{u} = \bm{u} $
        \item $\forall \bm{v} \in V\ \exists -\bm{u}\ \bm{u} + (-\bm{u}) = \bm{0}$
        \item $\forall \bm{u}, \bm{v} \in V\ \bm{u} + \bm{v} = \bm{v} + \bm{u}$
        \item $\forall a \in K\ \forall \bm{u}, \bm{v} \in V\ a(\bm{u} + \bm{v}) = a\bm{u} + a\bm{v}$
        \item $\forall a, b \in K\ \forall \bm{v} \in V\ (a + b)\bm{v} = a\bm{v} + b\bm{v}$
        \item $\forall a, b \in K\ \forall \bm{v} \in V\ a(b\bm{v}) = (ab)\bm{v}$
        \item $\exists 1 \in K\ \forall \bm{v} \in V\ 1\bm{v} = \bm{v}$
      \end{enumerate}
    \end{itembox}
    $\r{Herm}(N)$は通常の行列の演算規則に従えば,明らかに$\mathbb{C}$上のベクトル空間である.
    \par
    内積は,ベクトル空間$V$に対して定義された演算$(\cdot, \cdot): V\times V \to \mathbb{C}$が以下の性質を満たすものである.
    ただし,$\bm{u}, \bm{v}, \bm{w} \in V$,$a, b \in \mathbb{C}$とする.
    \begin{itembox}[l]{内積の定義}
      \begin{enumerate}
        \item $(\bm{u}, \lambda \bm{v}) = \lambda (\bm{u}, \bm{v})$
        \item $(\bm{u}, \bm{v}) = (\bm{v}, \bm{u})^*$
        \item $\forall \bm{u}\ (\bm{u}, \bm{u}) \geq 0$
        \item $(\bm{u}, \bm{u}) = 0 \implies \bm{u} = \bm{0}$
      \end{enumerate}
    \end{itembox}
    $A, B\in \r{Herm}(N)$に対して内積を定義するには,対角和を用いて,
    \begin{align}
      (A, B) = \tr(A^{\dag}B)
    \end{align}
    と定義すればよい.
    行列の対角和が$\r{Herm}(N)$の内積になることは非自明なので示す.
    ただし,$A, B\in \r{Herm}(N)$$A$の固有値を$\lambda_1, \dots \lambda_N$とする.
    \begin{proof}
      \begin{enumerate}
        \item $(A, \lambda B) = \tr\qty(A^{\dag}\lambda B) = \lambda\tr\qty(A^{\dag}B) = \lambda(A, B)$
        \item $(B, A)^* = \tr\qty(B^{\dag}A)^* = \tr\qty(\qty(B^{\dag}A)^{\dag}) = \tr{A^{\dag}B} = (A, B)$
        \item $A$はHermite行列であるから固有値は全て実数で,$(A, A) = \tr{A^{\dag}A} = \tr{A^2} = \sum_{i = 1}^{N}\lambda^2_i \geq 0$
        \item $(A, A) = \sum_{i = 1}^{N}\lambda^2_i = 0$となるのは$\lambda_1 = \cdots = \lambda_N = 0$となるときのみで,そのときは$A$は0行列である.
      \end{enumerate}
      よって,対角和を用いて内積を定義すると,$\r{Herm}(N)$は内積空間になる.
    \end{proof}
    Pauli行列を$\sqrt{2}$で割ったものは$\rm{Herm}(2)$の正規直交基底となる.
    ただし,Pauli行列は,
    \begin{align}
      \hat{\sigma}^0 &\coloneqq \mqty(1 & 0 \\ 0 & 1) \\ 
      \hat{\sigma}^1 &\coloneqq \mqty(0 & 1 \\ 1 & 0) \\ 
      \hat{\sigma}^2 &\coloneqq \mqty(0 & -\i \\ \i & 0) \\ 
      \hat{\sigma}^3 &\coloneqq \mqty(1 & 0 \\ 0 & -1)
    \end{align}
    と定義される.
    まず,Pauli行列を$\sqrt{2}$で割ったものが$\r{Herm}(2)$の基底であることを示す.
    \begin{proof}
      $a_0, a_1, a_2, a_3 \in \mathbb{C}$を用いると,
      \begin{align}
        \frac{1}{\sqrt{2}}a_0\hat{\sigma}^0 + \frac{1}{\sqrt{2}}a_1\hat{\sigma}^1 + \frac{1}{\sqrt{2}}a_2\hat{\sigma}^2 + \frac{1}{\sqrt{2}}a_3\hat{\sigma}^3 = \frac{1}{\sqrt{2}}\mqty(a_0 + a_3 & a_1 - \i a_2 \\ a_1 + \i a_2 & a_0 - a_3) 
      \end{align}
      となり,Pauli行列を$\sqrt{2}$で割ったものの線形結合で任意のHermite行列が書けることが分かる.
    \end{proof}
    \par
    また,Pauli行列を$\sqrt{2}$で割ったものは正規直交基底を成すことが分かる.
    \begin{proof}
      Pauli行列同士の内積について,
      \begin{align}
        \qty(\frac{1}{\sqrt{2}}\hat{\sigma}^0, \frac{1}{\sqrt{2}}\hat{\sigma}^0) &= \frac{1}{2}\tr\qty{\mqty(1 & 0 \\ 0 & 1)\mqty(1 & 0 \\ 0 & 1)} = \frac{1}{2}\tr{\mqty(1 & 0 \\ 0 & 1)} = 1 \\ 
        \qty(\frac{1}{\sqrt{2}}\hat{\sigma}^1, \frac{1}{\sqrt{2}}\hat{\sigma}^1) &= \frac{1}{2}\tr\qty{\mqty(0 & 1 \\ 1 & 0)\mqty(0 & 1 \\ 1 & 0)} = \frac{1}{2}\tr{\mqty(1 & 0 \\ 0 & 1)} = 1 \\ 
        \qty(\frac{1}{\sqrt{2}}\hat{\sigma}^2, \frac{1}{\sqrt{2}}\hat{\sigma}^2) &= \frac{1}{2}\tr\qty{\mqty(0 & -\i \\ \i & 0)\mqty(0 & -\i \\ \i & 0)} = \frac{1}{2}\tr{\mqty(1 & 0 \\ 0 & 1)} = 1 \\ 
        \qty(\frac{1}{\sqrt{2}}\hat{\sigma}^3, \frac{1}{\sqrt{2}}\hat{\sigma}^3) &= \frac{1}{2}\tr\qty{\mqty(1 & 0 \\ 0 & -1)\mqty(1 & 0 \\ 0 & -1)} = \frac{1}{2}\tr{\mqty(1 & 0 \\ 0 & 1)} = 1 \\ 
        \qty(\frac{1}{\sqrt{2}}\hat{\sigma}^0, \frac{1}{\sqrt{2}}\hat{\sigma}^1) &= \frac{1}{2}\tr\qty{\mqty(1 & 0 \\ 0 & 1)\mqty(0 & 1 \\ 1 & 0)} = \frac{1}{2}\tr{\mqty(0 & 1 \\ 1 & 0)} = 0 \\ 
        \qty(\frac{1}{\sqrt{2}}\hat{\sigma}^0, \frac{1}{\sqrt{2}}\hat{\sigma}^2) &= \frac{1}{2}\tr\qty{\mqty(1 & 0 \\ 0 & 1)\mqty(0 & -\i \\ \i & 0)} = \frac{1}{2}\tr{\mqty(0 & -\i \\ \i & 0)} = 0 \\ 
        \qty(\frac{1}{\sqrt{2}}\hat{\sigma}^0, \frac{1}{\sqrt{2}}\hat{\sigma}^3) &= \frac{1}{2}\tr\qty{\mqty(1 & 0 \\ 0 & 1)\mqty(1 & 0 \\ 0 & -1)} = \frac{1}{2}\tr{\mqty(1 & 0 \\ 0 & -1)} = 0 \\
        \qty(\frac{1}{\sqrt{2}}\hat{\sigma}^1, \frac{1}{\sqrt{2}}\hat{\sigma}^2) &= \frac{1}{2}\tr\qty{\mqty(0 & 1 \\ 1 & 0)\mqty(0 & -\i \\ \i & 0)} = \frac{1}{2}\tr{\mqty(\i & 0 \\ 0 & -\i)} = 0 \\ 
        \qty(\frac{1}{\sqrt{2}}\hat{\sigma}^1, \frac{1}{\sqrt{2}}\hat{\sigma}^3) &= \frac{1}{2}\tr\qty{\mqty(0 & 1 \\ 1 & 0)\mqty(1 & 0 \\ 0 & -1)} = \frac{1}{2}\tr{\mqty(0 & 1 \\ -1 & 0)} = 0 \\ 
        \qty(\frac{1}{\sqrt{2}}\hat{\sigma}^2, \frac{1}{\sqrt{2}}\hat{\sigma}^3) &= \frac{1}{2}\tr\qty{\mqty(0 & -\i \\ \i & 0)\mqty(1 & 0 \\ 0 & -1)} = \frac{1}{2}\tr{\mqty(0 & \i \\ \i & 0)} = 0 
      \end{align}
      となり,Pauli行列は正規直交基底であると分かる.
      内積の順序入れ替えについては,以上の計算結果が全て実数であることから省略する.
    \end{proof}
  \chapter{ソースコード}
    \inputminted[linenos, frame=lines, label=qst-code]{python}{/home/hawk/desktop/lab/b4_seminar/experiment/qst/main.py}
\end{document}