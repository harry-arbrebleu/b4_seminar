\documentclass{report}
\usepackage{luatexja}
\usepackage{tikz}
\usepackage{amsmath, amssymb, type1cm, amsfonts, latexsym, mathtools, bm, amsthm, url, color}
\usepackage{multirow, hyperref, longtable, dcolumn, tablefootnote, empheq}
\usepackage{tabularx, footmisc, colortbl, here, usebib, microtype, physics}
\usepackage{graphicx, luacode, caption, fancyhdr, appendix}
\usepackage{fancybox, color, tcolorbox, physics}
\usepackage[top = 20truemm, bottom = 20truemm, left = 20truemm, right = 20truemm]{geometry}
\usepackage{ascmac, fancybox, color, tabularray, subcaption}
\usepackage{multicol, siunitx}
\usepackage{luatexja-fontspec, ascmac, fancybox, pdfpages}
\usepackage{upgreek, colortbl, mhchem}
\usepackage{biblatex, array, truthtable}
\usepackage{listings, jvlisting}
\usepackage{xcolor, siunitx, float, dcolumn}
\usepackage{titlesec}
\usepackage{minted}

\titleformat{\section}[block]{\normalsize}{\LARGE{\textbf{第\thesection 回授業}}}{0.5em}{}
\sisetup{
  table-format = 1.5,
  table-number-alignment = center,
}
\newcounter{problem}[section]
\renewcommand*\oldstylenums[1]{\textosf{#1}}
% \abovedisplayskip = 0pt
% \belowdisplayskip = 0pt

\allowdisplaybreaks
\newcolumntype{t}{!{\vrule width 0.1pt}}
\newcolumntype{b}{!{\vrule width 1.5pt}}
\UseTblrLibrary{amsmath, booktabs, counter, diagbox, functional, hook, html, nameref, siunitx, varwidth, zref}
\setlength{\columnseprule}{0.4pt}
\captionsetup[figure]{font = bf}
\captionsetup[table]{font = bf}
\captionsetup[lstlisting]{font = bf}
\captionsetup[subfigure]{font = bf, labelformat = simple}
\setcounter{secnumdepth}{5}
\newcolumntype{d}{D{.}{.}{5}}
\newcolumntype{M}[1]{>{\centering\arraybackslash}m{#1}}
\everymath{\displaystyle}

\renewcommand{\figurename}{図}
\renewcommand{\tablename}{表}
\renewcommand{\theproblem}{\thesection-\arabic{problem}}
\renewcommand{\i}{\mathrm{i}}
\renewcommand{\laplacian}{\Delta}
\newcommand{\NOT}[1]{\overline{#1}}
\renewcommand{\hat}[1]{\overhat{#1}}
\renewcommand{\thesubfigure}{(\alph{subfigure})}
\newcommand{\m}[3]{\multicolumn{#1}{#2}{#3}}
\renewcommand{\r}[1]{\mathrm{#1}}
\newcommand{\e}{\r{e}}
\newcommand{\Ef}{E_{\r{F}}}
\renewcommand{\c}{\si{\degreeCelsius}}
\renewcommand{\d}{\r{d}}
\renewcommand{\t}[1]{\texttt{#1}}
\newcommand{\kb}{k_{\r{B}}}
\renewcommand{\phi}{\varphi}
\newcommand{\reff}[1]{\textbf{図\ref{#1}}}
\newcommand{\reft}[1]{\textbf{表\ref{#1}}}
\newcommand{\refe}[1]{\textbf{式\eqref{#1}}}
\newcommand{\refp}[1]{\textbf{ソースコード\ref{#1}}}
\newcommand{\refa}[1]{\textbf{\ref{#1}}}
\newcommand{\probbref}[1]{{\bfseries\sffamily 基礎問題\ref{problemb-ref:#1}}}
\newcommand{\probaref}[1]{{\bfseries\sffamily 発展問題\ref{problem-ref:#1}}}
\renewcommand{\lstlistingname}{ソースコード}
\renewcommand{\theequation}{\thesection.\arabic{equation}}
\renewcommand{\thefigure}{\thesection.\arabic{figure}}
\renewcommand{\thetable}{\thesection.\arabic{table}}
% \renewcommand{\thelstlisting}{\thesection.\arabic{lstlisting}}
% \renewcommand{\the}{def}
\renewcommand{\footrulewidth}{0.4pt}
\newcommand{\mar}[1]{\textcircled{\scriptsize #1}}
\newcommand{\combination}[2]{{}_{#1} \mathrm{C}_{#2}}
\newcommand{\thline}{\noalign{\hrule height 0.1pt}}
\newcommand{\bhline}{\noalign{\hrule height 1.5pt}}
\renewcommand{\thesection}{\arabic{section}}
\renewcommand{\appendixname}{付録}

\pagestyle{fancy}

\chead{B4新人研修}
\rhead{}
\cfoot{\thepage}
\lhead{}
\rfoot{62200139\ 青木\ 陽}
\setcounter{tocdepth}{4}
\makeatletter
\@addtoreset{equation}{subsection}
\makeatother

\title{偏光を用いた2量子もつれの量子状態トモグラフィによる評価}
\date{\today}
\author{62200139\ 青木\ 陽}
\addbibresource{ref.bib}
\defbibheading{bunken}[\refname]{\section*{#1}}
\begin{document}
  以降では,複素数であることを強調するとき,チルダを付けることにする.
  \subsection{偏光状態}
    真空中のMaxwell方程式は,
    \begin{align}
      \begin{dcases}
        \curl \bm{E} = -\pdv{\bm{B}}{t} \\ 
        \curl \bm{B} = \mu_0\epsilon_0\pdv{\bm{E}}{t} \\ 
        \div \bm{E} = 0 \\ \label{maxwell-vacuume}
        \div \bm{B} = 0
      \end{dcases}
    \end{align}
    である.
    微分方程式は線型であるから,以降では,$\bm{E} \to \tilde{\bm{E}}$,$\bm{B} \to \tilde{\bm{B}}$とする.
    なお,実際の物理量としての電場や磁場は実数であるから,
    \begin{align}
      \bm{E} &= \Re\qty(\tilde{\bm{E}}) \\ 
      \bm{B} &= \Re\qty(\tilde{\bm{B}})
    \end{align}
    とする.
    さて,\refe{maxwell-vacuume}の第1式の両辺に$\curl$を作用させると,
    \begin{align}
      \curl\qty(\curl \tilde{\bm{E}}) &= -\pdv{t}\qty(\curl{\tilde{\bm{B}}}) \\ 
      \iff \grad\qty(\div\tilde{\bm{E}}) - \laplacian\tilde{\bm{E}} &= -\mu_0\epsilon_0\pdv[2]{\tilde{\bm{E}}}{t} \label{electro-field-wave-1}\\ 
      \iff \laplacian\tilde{\bm{E}} &= \mu_0\epsilon_0\pdv[2]{\tilde{\bm{E}}}{t} \label{electro-field-wave-2} \\ 
    \end{align}
    となる.
    \refe{electro-field-wave-1}から\refe{electro-field-wave-2}の式変形で\refe{maxwell-vacuume}の第3式を用いた.
    磁場に関しても同様に計算をすると,
    \begin{align}
      \laplacian\tilde{\bm{B}} &= \mu_0\epsilon_0\pdv[2]{\tilde{\bm{B}}}{t} \label{magnetic-field-wave-2} \\ 
    \end{align}
    となる.
    \par
    さて,\refe{electro-field-wave-2}や\refe{magnetic-field-wave-2}なる偏微分方程式の解の1つに,
    \begin{align}
      \tilde{\bm{E}} = \tilde{\bm{E}_0}\e^{\i(\bm{k}\cdot \bm{r} - \omega t)} \label{electro-field-wave-3} \\ 
      \tilde{\bm{B}} = \tilde{\bm{B}_0}\e^{\i(\bm{k}\cdot \bm{r} - \omega t)} \label{magnetic-field-wave-3}
    \end{align}
    がある.
    \refe{electro-field-wave-3}と\refe{magnetic-field-wave-3}を\refe{maxwell-vacuume}の第2式に代入すると,
    \begin{align}
      \curl \tilde{\bm{B}_0}\e^{\i(\bm{k}\cdot \bm{r} - \omega t)} &= \mu_0\epsilon_0\pdv{t}\tilde{\bm{E}_0}\e^{\i(\bm{k}\cdot \bm{r} - \omega t)} \\ 
      \i\bm{k}\times\tilde{\bm{B}} &= -\i\omega\mu_0\epsilon_0\tilde{\bm{E}} \\ 
      \bm{k}\times\tilde{\bm{B}} &= -\omega\mu_0\epsilon_0\tilde{\bm{E}} \label{electro-magnetic-wave-vertical}
    \end{align}
    となる.
    \refe{electro-magnetic-wave-vertical}の両辺で$\bm{k}$との内積を取ると,
    \begin{align}
      \bm{k}\times\qty(\bm{k}\times\tilde{\bm{B}}) &= -\omega\mu_0\epsilon_0\bm{k}\cdot\tilde{\bm{E}} \\ 
      \iff \tilde{\bm{B}}\qty(\bm{k}\times\bm{k}) &= -\omega\mu_0\epsilon_0\bm{k}\cdot\tilde{\bm{E}} \\ 
      \iff \bm{k}\cdot \bm{E} &= 0
    \end{align}
    となり,電場$\tilde{\bm{E}}$と波数ベクトル$\bm{k}$が直交することが分かる.
    磁場についても\refe{maxwell-vacuume}を用いれば,同様に波数ベクトルと直交することが分かる.
    よって,波の進行方向を$z$方向と定義しても一般性を失わないので,
    \begin{align}
      \bm{E} &= \mqty(E_x \\ E_y \\ 0) \\ 
      &= \mqty(E_{x0}\cos(kz - \omega t) \\ E_{y0}\cos(kz - \omega t + \phi) \\ 0) \label{ex-ey-representation}
    \end{align}
    とする.
    \par
    いよいよ,具体的な偏光を考える.
    \refe{ex-ey-representation}より,
    \begin{align}
      \begin{dcases}
        \frac{E_x}{E_{x0}} = \cos(kz - \omega t) \\ 
        \frac{E_y}{E_{y0}} = \cos(kz - \omega t + \phi)
      \end{dcases}
    \end{align}
    と書ける.
    和積の公式を用いると,
    \begin{align}
      \qty(\frac{E_{y}}{E_{y0}}) - \qty(\frac{E_x}{E_{x0}})\cos\phi = -\sin(kz - \omega t)\sin\phi
    \end{align}
    となる.
    両辺を2乗すると,
    \begin{align}
      \qty(\frac{E_{y}}{E_{y0}})^2 - 2\qty(\frac{E_x}{E_{x0}})\qty(\frac{E_{y}}{E_{y0}})\cos\phi + \qty(\frac{E_x}{E_{x0}})^2\cos^2\phi &= \qty(1 - \sin^2(kz - \omega t))\sin^2\phi \\ 
      \qty(\frac{E_{x}}{E_{x0}})^2 - 2\qty(\frac{E_x}{E_{x0}})\qty(\frac{E_{y}}{E_{y0}})\cos\phi + \qty(\frac{E_y}{E_{y0}})^2&= \sin^2\phi \label{elliptical-polarization}
    \end{align}
    となる.
    一般に,この2次曲線は楕円を描くことが知られている.
    以下では,2つの特別な場合を考える.
    \begin{enumerate}
      \item $\phi = m\pi\ m \in \mathbb{Z}$のとき.
        このとき,\refe{elliptical-polarization}は,
        \begin{align}
          &\qty(\frac{E_x}{E_{x0}} - (-1)^m\frac{E_y}{E_{y0}})^2 = 0 \\ 
          &\implies \frac{E_x}{E_{x0}} = (-1)^m\frac{E_y}{E_{y0}}
        \end{align}
        となり,$m$が偶数のときは$E_x$と$E_y$は同位相,$m$が奇数のときは$\pi$だけ位相がずれて振動することが分かる.
        特に$E_x = E_y$であれば,$m$が偶数であることを$\ang{45}$偏光,$m$が奇数であることを$-\ang{45}$偏光という.
      \item $\phi = \qty(m + \frac{1}{2})\pi\ m \in \mathbb{Z}$のとき.
        このとき,\refe{elliptical-polarization}は,
        \begin{align}
          \qty(\frac{E_x}{E_{x0}})^2 + \qty(\frac{E_y}{E_{y0}})^2 = 1
        \end{align}
        となり,$E_x / E_{x0}$と$E_y / E_{y0}$は単位円周上にあるという関係をもつので,円偏光であると分かる.
        特に$E_x = E_y$であれば,$m$が偶数のであることを$z$軸正から負方向に見て左周りに回るので左偏光,$m$が奇数のときは右偏光という.
    \end{enumerate}
  \subsection{偏光行列,Stokesパラメータ,Poincar\'e球}
    前小節までで,真空中を伝搬する電磁場の偏光を記述する方法を示した.
    本小節では他に,偏光を表す方法を説明する.
    \par
    1つのモードの偏光行列は以下のように定義される.
    \begin{align}
      S &\coloneqq \mqty(\tilde{E}_x \\ \tilde{E}_y)\mqty(\tilde{E}^*_x & \tilde{E}^*_y) \\ 
      &= \mqty(\abs{\tilde{E}_x}^2 & \tilde{E}_x\tilde{E}^*_y \\ \tilde{E}^*_x\tilde{E}_y & \abs{\tilde{E}_y}^2)
    \end{align}
    多モードの偏光行列は,
    \begin{align}
      S &\coloneqq \mqty(\ev{\abs{\tilde{E}_x}}^2 & \ev{\tilde{E}_x\tilde{E}^*_y} \\ \ev{\tilde{E}^*_x\tilde{E}_y} & \ev{\abs{\tilde{E}_y}^2})
    \end{align}
    となる.
    ただし,$\ev{\cdot}$は長時間平均で,$\sigma$をモードのラベルとすると,
    \begin{align}
      \ev{f(x)} \coloneqq \lim_{T \to \infty} \frac{1}{T}\int_{0}^{T}\dd{t}\sum_{\sigma}f_{\sigma}(t)
    \end{align}
    である.
    \par
    再び,単一モードの偏光を考える.
    偏光行列の対角成分が複素共役であることに注意すれば,$S$はHermite行列であることが分かる.
    \ref{pauli-matrix}での議論を踏まえれば,Pauli行列は$2\times 2$のHermite行列の基底であるから,
    \begin{align}
      S = \sum_{i}\check{S}_i\hat{\sigma}^i
    \end{align}
    と展開できる.
    内積は対角和で定義されているのだから,
    \begin{align}
      \check{S}_i &= \frac{\tr\qty{\hat{\sigma}^iS}}{\tr\qty{\qty(\hat{\sigma}^i)^2}} \\ 
      &= \frac{1}{2}\tr\qty{\hat{\sigma}^iS}
    \end{align}
    と書ける.
    具体的に4成分を求めると,
    \begin{align}
      \check{S}_0 &= \frac{1}{2}\qty(\abs{\tilde{E}_x}^2 + \abs{\tilde{E}_y}^2) \\ 
      \check{S}_1 &= \frac{1}{2}\qty(\tilde{E}^*_x\tilde{E}_y + \tilde{E}_x\tilde{E}^*_y ) = \Re\qty(\tilde{E}^*_x\tilde{E}_y)\\ 
      \check{S}_2 &= \frac{1}{2}\qty(-\i\tilde{E}_x\tilde{E}^*_y + \i\tilde{E}^*_x\tilde{E}_y) = \Im\qty(\tilde{E}^*_x\tilde{E}_y)\\ 
      \check{S}_3 &= \frac{1}{2}\qty(\abs{\tilde{E}_x}^2 - \abs{\tilde{E}_y}^2) 
    \end{align}
    古典光学の慣習に従ってStokesパラメータを定義すると,
    \begin{align}
      S_0 &\coloneqq 2\check{S}_0 = \abs{\tilde{E}_x}^2 + \abs{\tilde{E}_y}^2 \\ 
      S_1 &\coloneqq 2\check{S}_3 = \abs{\tilde{E}_x}^2 - \abs{\tilde{E}_y}^2 \\ 
      S_2 &\coloneqq 2\check{S}_1 = 2\Re\qty(\tilde{E}^*_x\tilde{E}_y) \\ 
      S_3 &\coloneqq 2\check{S}_2 = 2\Im\qty(\tilde{E}^*_x\tilde{E}_y)
    \end{align}
    となる.
    Stokesパラメータには,
    \begin{align}
      S^2_1 + S^2_2 + S^2_3 = S^2_0 \label{stokes-parameter-relation}
    \end{align}
    なる関係が成立する.
    \begin{proof}
      \begin{align}
        S^2_1 + S^2_2 + S^2_3 &= 4\Re\qty(\tilde{E}^*_x\tilde{E}_y)^2 + 4\Im\qty(\tilde{E}^*_x\tilde{E}_y)^2 + \qty(\abs{\tilde{E}_x}^2 - \abs{\tilde{E}_y}^2)^2 \\ 
        &= 4\abs{\tilde{E}^*_x\tilde{E}_y}^2 + \abs{\tilde{E}_x}^4 + \abs{\tilde{E}_y}^4 - 2\tilde{E}^*_x\tilde{E}_y\tilde{E}_x\tilde{E}^*_y \\ 
        &= \abs{\tilde{E}_x}^4 + \abs{\tilde{E}_y}^4 + 2\abs{\tilde{E}_x}^2\abs{\tilde{E}_y}^2 \\ 
        &= \qty(\abs{\tilde{E}_x}^2 + \abs{\tilde{E}_y}^2)^2 \\ 
        &= S^2_0
      \end{align}
    \end{proof}
    \refe{stokes-parameter-relation}より,Stokesパラメータの自由度は実質的に3成分であるから,3次元空間で表すことができる.
    これが,Poincar\'e球である.
    \reft{stokes-parameter-relation-table}としてStokesパラメータ\footnote{
      2準位系のBloch球では,
      \begin{align}
        u &= 2\Re\qty(\rho^{(-\omega)}_{\r{ba}}) \\ 
        v &= 2\Im\qty(\rho^{(-\omega)}_{\r{ba}}) \\ 
        w &= \rho_{\r{bb}} - \rho_{\r{aa}}
      \end{align}
      なる関係がある.
      やはり,Stokesパラメータのインデックスはおかしいと思われる.
    }と偏光状態の関係を示す.
    \begin{table}[H]
      \centering
      \caption{Stokesパラメータと偏光状態の関係}\label{stokes-parameter-relation-table}
      \begin{tabular}{cc}
        \bhline
        Stokesパラメータ & 偏光状態 \\ \hline
        $S_1 = 1$ & $x$偏光 \\ 
        $S_1 = -1$ & $y$偏光 \\ 
        $S_2 = 1$ & $\ang{45}$偏光 \\ 
        $S_2 = -1$ & $-\ang{45}$偏光 \\ 
        $S_3 = 1$ & 左回り偏光 \\ 
        $S_3 = -1$ & 右回り偏光 \\ \bhline
      \end{tabular}
    \end{table}
  \subsection{複屈折}
    物質中のMaxwell方程式は,
    \begin{align}
      \begin{dcases}
        \curl \bm{E} = -\pdv{\bm{B}}{t} \\ 
        \curl \bm{B} = \mu_0\pdv{\bm{D}}{t} \\ 
        \div \bm{D} = 0 \\ \label{maxwell-material}
        \div \bm{B} = 0
      \end{dcases}
    \end{align}
    である.
    強磁性体でないため,$\mu = \mu_0$とした.
    \refe{maxwell-material}において,
    \begin{align}
      \bm{D} = \epsilon\bm{E}
    \end{align}
    なる関係があるとする.
    ただし,$\epsilon$は2階のテンソルであるとする.
    以下では,結晶の対称性を仮定して,
    \begin{align}
      \epsilon = \mqty(\dmat[0]{n^2_x, n^2_y, n^2_z})
    \end{align}
    とする.
    電場が,
    \begin{align}
      \tilde{\bm{E}} = \tilde{\bm{E}_0}\e^{\i(\bm{k}\cdot \bm{r} - \omega t)} 
    \end{align}
    と書けるとする.
    また,波数ベクトル$\bm{k}$が,
    \begin{align}
      \bm{k} = \mqty(k_x \\ k_y \\ k_z)
    \end{align}
    と書けるとする.
    真空状態でのMaxwell方程式の解析と同じように,$\bm{E} \to \tilde{\bm{E}}$,$\bm{B} \to \tilde{\bm{B}}$とする.
    \refe{maxwell-material}の第1式の両辺に$\curl$を作用させると,
    \begin{align}
      &\curl\curl\tilde{\bm{E}} = -\pdv{t}\qty(\curl\tilde{\bm{B}}) \\ 
      \iff &\grad\qty(\div\tilde{\bm{E}}) - \laplacian\tilde{\bm{E}} = \mu\pdv[2]{t}\qty(\epsilon\tilde{\bm{E}}) \\ 
      \iff &\qty(\bm{k}\cdot \tilde{\bm{E}}) - (\bm{k}\cdot \bm{k})\tilde{\bm{E}} = -\mu_0\omega^2\epsilon\tilde{\bm{E}} \\ 
      \iff &\mqty(
        \qty(-k^2_y - k^2_z + \mu_0n^2_x\omega^2)\tilde{E}_x + k_xk_y\tilde{E}_y + k_xk_z\tilde{E}_z \\ 
        k_xk_y\tilde{E}_x + \qty(-k^2_x - k^2_z - \mu_0n^2_y\omega^2)\tilde{E}_y + k_yk_z\tilde{E}_z \\ 
        k_xk_z\tilde{E}_x + k_yk_z\tilde{E}_y + \qty(-k^2_x - k^2_y - \mu_0n^2_z\omega^2)\tilde{E}_z \\ 
      ) = \mqty(0 \\ 0 \\ 0) \\ 
      \iff &\mqty(
        \qty(-k^2_y - k^2_z + \mu_0n^2_x\omega^2) & k_xk_y & k_xk_z \\ 
        k_xk_y & \qty(-k^2_x - k^2_z + \mu_0n^2_y\omega^2) & k_yk_z \\ 
        k_xk_z &  k_yk_z & \qty(-k^2_x - k^2_y + \mu_0n^2_z\omega^2) \\ 
      )\mqty(\tilde{E}_x \\ \tilde{E}_y \\ \tilde{E}_z) = \mqty(0 \\ 0 \\ 0) \label{epsilon-matrix}
    \end{align}
    となる.
    $\tilde{\bm{E}} \neq \bm{0}$の非自明な解が存在する条件は,$\tilde{\bm{E}}$にかかる行列の行列式が0となることなので,
    \begin{align}
      &\mqty|
        \qty(-k^2_y - k^2_z + \mu_0n^2_x\omega^2) & k_xk_y & k_xk_z \\ 
        k_xk_y & \qty(-k^2_x - k^2_z + \mu_0n^2_y\omega^2) & k_yk_z \\ 
        k_xk_z &  k_yk_z & \qty(-k^2_x - k^2_y + \mu_0n^2_z\omega^2) \\ 
      | = 0 \\ 
      \iff &\qty(-k^2_y - k^2_z + \mu_0n^2_x\omega^2)\qty(-k^2_x - k^2_z + \mu_0n^2_y\omega^2)\qty(-k^2_x - k^2_y + \mu_0n^2_z\omega^2) + 2k^2_xk^2_yk^2_z \notag \\
      &- k^2_xk^2_z\qty(-k^2_x - k^2_z + \mu_0n^2_y\omega^2) - k^2_xk^2_y\qty(-k^2_x - k^2_y + \mu_0n^2_z\omega^2) - k^2_yk^2_z\qty(-k^2_y - k^2_z + \mu_0n^2_x\omega^2) = 0 \\ 
      \iff &\mu^3_0n^2_xn^2_yn^2_z\omega^6 - \mu^2_0\qty{\qty(k^2_y + k^2_z)n^2_yn^2_z + \qty(k^2_x + k^2_z)n^2_xn^2_z +\qty(k^2_x + k^2_y)n^2_xn^2_y}\omega^4 + \mu_0\qty(k^2_x + k^2_y + k^2_z)\qty(k^2_xn^2_x + k^2_yn^2_y + k^2_zn^2_z)\omega^2 = 0 \\ 
      \iff &\mu^2_0\omega^4 - \mu_0\qty(\frac{k^2_x + k^2_y}{n^2_z} + \frac{k^2_y + k^2_z}{n^2_x} + \frac{k^2_z + k^2_x}{n^2_y})\omega^2 + \qty(k^2_x + k^2_y + k^2_z)\qty(\frac{k^2_x}{n^2_yn^2_z} + \frac{k^2_y}{n^2_zn^2_x} + \frac{k^2_z}{n^2_xn^2_y}) = 0 \label{o-e-eq}
    \end{align}
    となる.
    \refe{o-e-eq}を$\mu\omega^2$について解くと,
    \begin{align}
      \mu_0\omega^2 &= \qty(\frac{k^2_x + k^2_y}{n^2_z} + \frac{k^2_y + k^2_z}{n^2_x} + \frac{k^2_z + k^2_x}{n^2_y}) \notag\\
      &\pm \sqrt{\frac{-3k^2_xk^2_z - 3k^2_yk^2_z - 3k^4_z + k^2_xk^2_y}{n^2_xn^2_y} + \frac{-3k^2_yk^2_x - 3k^2_zk^2_x - 3k^4_x + k^2_yk^2_z}{n^2_yn^2_z} + \frac{-3k^2_zk^2_y - 3k^2_xk^2_y - 3k^4_y + k^2_zk^2_x}{n^2_zn^2_x}} \label{mu-omega-solution}
    \end{align}
    となる.
    $n_x = n_y = n_{\r{o}}$,$n_z = n_{\r{e}}$のとき,
    \refe{mu-omega-solution}は,% 省略
    \begin{align}
      \mu_0\omega^2 = \frac{k^2_x}{n^0_{\r{o}}} + \frac{k^2_y}{n^0_{\r{o}}} + \frac{k^2_z}{n^0_{\r{o}}}, \frac{k^2_x}{n^0_{\r{e}}} + \frac{k^2_y}{n^0_{\r{e}}} + \frac{k^2_z}{n^0_{\r{o}}}
    \end{align}
    となり,2種類の波数ベクトルが許容されることになる.
  \subsection{波長板の効果}
\end{document}